\documentclass[a4paper,xcolor=dvipsnames]{book}

\usepackage[utf8]{inputenc}
\usepackage[T1]{fontenc}
\usepackage[ngerman]{babel}
\usepackage[a4paper]{geometry}
\usepackage{parskip}
\usepackage{graphicx}
\usepackage{amsmath}
\usepackage{amssymb}
\usepackage{amsthm}
\usepackage{cancel}
\usepackage{booktabs}
\usepackage[space]{grffile}
\usepackage{verbatim}
\usepackage{changepage}
\usepackage{marvosym}
\usepackage[dvipsnames]{xcolor}
\usepackage[hidelinks,breaklinks]{hyperref}
\usepackage{subcaption}
\usepackage{float}
\usepackage[normalem]{ulem}
\usepackage{csquotes}
\usepackage{sectsty}

\usepackage{fontspec}
\setmainfont[Ligatures=TeX,
             Extension=.ttf,
             Path=../../karmilla/ttf/]{Karmilla-Regular-016}

\definecolor{sblau}{HTML}{007ec1}
\allsectionsfont{\color{sblau}}

\theoremstyle{plain}
\newtheorem{theorem}{Satz}[chapter]

\theoremstyle{definition}
\newtheorem{definition}{Definition}[chapter]
\newtheorem{exercise}{Übung}[chapter]
\newtheorem{solution}{Lösung}[chapter]
\newtheorem{alternativeproof}{Alternativer Beweis}[chapter]
\newtheorem{warning}{Warnung}[chapter]

\theoremstyle{remark}
\newtheorem{hint}{Hinweis}[chapter]
\newtheorem{explanation}{Erklärung}[chapter]
\newtheorem{proofsummary}{Beweiszusammenfassung}[chapter]
\newtheorem{solutionprocess}{Lösungsweg}[chapter]
\newtheorem{example}{Beispiel}[chapter]
\newtheorem{question}{Frage}[chapter]
\newtheorem{answer}{Antwort}[chapter]

\errorcontextlines 10000

\newenvironment{importantparagraph}{\begin{adjustwidth}{1.5cm}{}}{\end{adjustwidth}}
\newenvironment{indentblock}{\begin{adjustwidth}{.5cm}{}}{\end{adjustwidth}}

\definecolor{forestgreen}{rgb}{0.13, 0.55, 0.13}

\makeatletter
\renewenvironment{proof}[1][\relax]{\par
  \pushQED{\qed}%
  \trivlist
  \item[\hskip\labelsep\itshape
      \ifx#1\relax \proofname\else\proofname{} (#1)\fi\@addpunct{.}]\ignorespaces
}{%
  \popQED\endtrivlist\@endpefalse
  \unskip
}
\makeatother

\begingroup
    \makeatletter
    \@for\theoremstyle:=definition,remark,plain\do{%
        \expandafter\g@addto@macro\csname th@\theoremstyle\endcsname{%
            \addtolength\thm@preskip\parskip
            }%
        }
\endgroup
